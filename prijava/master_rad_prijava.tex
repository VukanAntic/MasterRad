\documentclass[a4paper]{article}
\usepackage[utf8]{inputenc}
\usepackage[T2A]{fontenc}
\setlength{\textheight}{25cm}
\setlength{\textwidth}{18cm}
\setlength{\topmargin}{-25mm}
\setlength{\hoffset}{-25mm}
\def\zn{,\kern-0.09em,}

\begin{document}
\thispagestyle{empty}

\begin{flushleft}
Математички факултет\\
Универзитета у Београду
\end{flushleft}

\bigskip

\begin{center}
\textbf{МОЛБА\\
ЗА ОДОБРАВАЊЕ ТЕМЕ МАСТЕР РАДА
}\end{center}

\bigskip

\begin{flushleft}
Молим да ми се одобри израда мастер рада под насловом:
\end{flushleft}

\begin{minipage}{16.5cm}
\textbf{\textit{\zn Упоређивање протокола за комуникацију GraphQL, GRPC и Rest, развојом мобилне апликације употребом технологија Spring Boot и React Native''}}
\end{minipage}\\
\rule[4mm]{17.5cm}{.05mm}
\begin{flushleft}
\framebox{
\begin{minipage}[t][7cm]{17cm}
\textbf{Значај теме и области:}
\\
Са растућом количином мобилних апликација, комуникација између клијента и сервера се најчешће сведе на коришћење Rest протокола, али, протоколи као што су  GraphQL и GRPC дају боље преформансе, повећану флексибилност, као и скалабилност. Зато, имплементацијом свих наведених протокола у оквиру исте апликације може показати кад је најбоље користити који протокол.
\\
\textbf{Специфични циљ рада:}

\\
Циљ овог рада је поређење претходно наведених протокола за комуникацију између клијента и сервера. У ту сврху, биће развијена микросервисна апликација за Android и IOS, коришћењем Spring Boot и React Native технологија. Апликација ће свакодневно давати кориснику нову уметничку слику, са детаљним описом, и даће му могућност да сачува своје омиљене, ради даљег препоручивања новог садржаја. Поред тога, корисник ће моћи да дели слике преко других апликација, као и да сачува слику у меморију свог телефона. Такође, при изласку нове слике за корисника, добиће обавештење на свом уређају, тако да не би пропустио новитете у апликацији. У оквиру рада ће се извршити поређење перформанси комуникације на развијеној апликацији, и биће изведени одговарајући закључци. Развијена апликација ће бити доступна као софтвер отвореног кода.
\end{minipage}
}
\end{flushleft}
\vspace{1cm}
\makebox[10cm][c]{\textbf{Вукан Антић, 1071/2022, Информатика}}
Сагласан ментор \makebox[5cm][c]{\textbf{Владимир Филиповић}} \\
\rule[4mm]{10cm}{.05mm} \hfill \raisebox{4mm}{\makebox[5cm][l]{.\dotfill.}} \\
\raisebox{1cm}%
[9mm][0mm]{\makebox[10cm][c]{\textit{(име и презиме студента, бр. индекса, модул)}}} \\
\makebox[10cm]{ }\\
\vspace{-1cm}\\
\rule[2cm]{6.5cm}{.05mm} \hfill \rule[2cm]{6.5cm}{.05mm}\\
\vspace{-2.4cm}\\
\raisebox{2cm}{\makebox[6.5cm][c]{\textit{(својеручни потпис студента)}}}
\hfill \raisebox{2cm}{\makebox[6.5cm][c]{\textit{(својеручни потпис ментора)}}}\\
\vspace{-2cm}\\
%%%%%%%%%%%%%%%%%%%%%%%%%%%%%%%%%%%%%%%%%%%%%%%%%%%%%%%%%%%%%%%%%%%%%%%%%%%%%%%
% u donji red uneti datum podnosenja molbe									  %
%%%%%%%%%%%%%%%%%%%%%%%%%%%%%%%%%%%%%%%%%%%%%%%%%%%%%%%%%%%%%%%%%%%%%%%%%%%%%%%
\makebox[5.5cm][c]{\textbf{21.4.2025.}}\makebox[5.5cm]{}  Чланови комисије\\
%%%%%%%%%%%%%%%%%%%%%%%%%%%%%%%%%%%%%%%%%%%%%%%%%%%%%%%%%%%%%%%%%%%%%%%%%%%%%%%
% POPUNJAVA MENTOR (rucno ili na sledeci nacin):							  %
% u donji red umesto .\dotfill. upisati podatke o 1. clanu komisije		      %
%%%%%%%%%%%%%%%%%%%%%%%%%%%%%%%%%%%%%%%%%%%%%%%%%%%%%%%%%%%%%%%%%%%%%%%%%%%%%%%
\rule[4mm]{5.5cm}{.05mm}\makebox[5.5cm]{ } 1. \makebox[6cm][l]{Александар Картељ}\\
\vspace{-8mm}\\
\raisebox{4mm}%														
[7mm][0mm]{\makebox[5.5cm][c]{\textit{(датум подношења молбе)}}}\makebox[5.5cm]{ }
%%%%%%%%%%%%%%%%%%%%%%%%%%%%%%%%%%%%%%%%%%%%%%%%%%%%%%%%%%%%%%%%%%%%%%%%%%%%%%%
% POPUNJAVA MENTOR (rucno ili na sledeci nacin): 							  %
% u donji red umesto .\dotfill. upisati podatke o 2. clanu komisije           %
%%%%%%%%%%%%%%%%%%%%%%%%%%%%%%%%%%%%%%%%%%%%%%%%%%%%%%%%%%%%%%%%%%%%%%%%%%%%%%%
2. \makebox[6cm][l]{Сташа Вујичић Станковић}\\

\vspace{1cm}


\begin{flushleft}
Катедра \makebox[9.5cm][l]{\textbf{за рачунарство и информатику}} је сагласна са предложеном темом.
\vspace{-3mm}
\hspace*{13mm} \rule[2.3cm]{9.5cm}{.05mm}\\
\vspace{-1cm}
%%%%%%%%%%%%%%%%%%%%%%%%%%%%%%%%%%%%%%%%%%%%%%%%%%%%%%%%%%%%%%%%%%%%%%%%%%%%%%
% POPUNJAVA SEF KATEDRE                                                      %
%%%%%%%%%%%%%%%%%%%%%%%%%%%%%%%%%%%%%%%%%%%%%%%%%%%%%%%%%%%%%%%%%%%%%%%%%%%%%%
\makebox[6.5cm][c]{} \hfill \makebox[6.5cm][c]{}\\
\rule[4mm]{6.5cm}{.05mm} \hfill \rule[4mm]{6.5cm}{.05mm}\\
\vspace{-5mm}
\makebox[6.5cm][c]{\textit{(шеф катедре)}} \hfill \makebox[6.5cm][c]{\textit{(датум одобравања молбе)}}
\end{flushleft}
\end{document} 